\documentclass{beamer}
\usepackage[utf8]{inputenc}
\usepackage{minted}
\usepackage{graphicx}
\usepackage{hyperref}
\usepackage{caption}
\usepackage{booktabs}

\usetheme{Madrid}
\usecolortheme{seahorse}
\setminted{fontsize=\scriptsize, breaklines=true}

\title[AlphaRTC Evaluation]{Evaluating Video Transmission Quality using AlphaRTC}
\author[X. Li et al.]{Xinnian Li, Zijun Peng, Bairui Li, Sijie Li, Zhang Zhang}
\institute{Peking University}
\date{May 2025}

\begin{document}

\frame{\titlepage}

\begin{frame}{Overview}
\begin{itemize}
    \item Use AlphaRTC for real-time video transmission.
    \item Evaluate video quality using VMAF.
    \item Estimate network throughput.
    \item Present results through visualization.
\end{itemize}
\end{frame}

\begin{frame}{AlphaRTC and VMAF}
\textbf{AlphaRTC:}
\begin{itemize}
    \item Open platform for real-time communication.
    \item Uses reinforcement learning for congestion control.
\end{itemize}

\textbf{VMAF:}
\begin{itemize}
    \item Open-source perceptual video quality metric.
    \item Developed by Netflix.
\end{itemize}
\end{frame}

\begin{frame}[fragile]{Video Preprocessing (1)}
\textbf{Step 1: Resize video}
\begin{minted}{console}
ffmpeg -i usain_bolt.mp4 -vf "scale=320:240,setdar=4/3" usain_bolt_0.mp4
\end{minted}

\textbf{Step 2: Crop 11s–26s}
\begin{minted}{console}
ffmpeg -ss 00:00:11 -to 00:00:26 -i usain_bolt_0.mp4 usain_bolt_1.mp4
\end{minted}
\end{frame}

\begin{frame}[fragile]{Video Preprocessing (2)}
\textbf{Step 3: Reduce FPS to 10}
\begin{minted}{console}
ffmpeg -i usain_bolt_1.mp4 -filter:v fps=10 usain_bolt_10.mp4
\end{minted}

\textbf{Step 4: Extract YUV and WAV}
\begin{minted}{console}
ffmpeg -i usain_bolt_10.mp4 usain_bolt.yuv
ffmpeg -i usain_bolt_10.mp4 usain_bolt.wav
\end{minted}
\end{frame}

\begin{frame}[fragile]{Modifying AlphaRTC Code}
\textbf{Customize BandwidthEstimator.py}
\begin{minted}{python}
class Estimator(object):
    def report_states(self, stats: dict):
        with open("stats.txt", "a") as f:
            f.write(str(stats) + "\n")

    def get_estimated_bandwidth(self)->int:
        return int(1e6) # 1Mbps
\end{minted}
\end{frame}

\begin{frame}[fragile]{Transmission Command}
\begin{minted}{console}
sudo docker run -d --rm \
  -v <path>/corpus:/app \
  -w /app --name alphartc \
  --platform linux/amd64 alphartc \
  peerconnection_serverless receiver_pyinfer.json

sudo docker exec alphartc \
  peerconnection_serverless sender_pyinfer.json
\end{minted}
\end{frame}

\begin{frame}[fragile]{Quality Evaluation with VMAF}
\begin{minted}{console}
vmaf --reference usain_bolt.yuv \
  --distorted out_usain_bolt.yuv \
  --width 320 --height 240 \
  --pixel_format 420 --bitdepth 8 \
  --model version=vmaf_v0.6.1 \
  --output output.xml --frame_cnt 30
\end{minted}
\begin{itemize}
    \item Compares first 30 frames after frame 24 for best score.
\end{itemize}
\end{frame}

\begin{frame}[fragile]{Throughput Estimation Code}
\begin{minted}{python}
total_time_ms = 0
total_payload_size = 0
stats = [eval(x) for x in open("stats.txt")]

for data in stats:
    if 'arrival_time_ms' in data:
        delta = data["arrival_time_ms"] - last_arrival_time
        total_time_ms += delta
        last_arrival_time = data["arrival_time_ms"]
        total_payload_size += int(data["payload_size"])

print(total_payload_size * 1000 / total_time_ms)
\end{minted}
\end{frame}

\begin{frame}{Results Summary}
\begin{itemize}
    \item \textbf{VMAF Score:} 8.27
    \item \textbf{Throughput:} 21.89 KB/ms
    \item VMAF plot saved and analyzed.
\end{itemize}

\begin{figure}
\includegraphics[width=0.7\textwidth]{vmaf.png}
\caption{VMAF Score per Frame}
\end{figure}
\end{frame}

\begin{frame}{Conclusion}
\begin{itemize}
    \item AlphaRTC + VMAF = efficient evaluation of video quality.
    \item Preprocessing and frame selection are key.
    \item Throughput and VMAF provide clear performance metrics.
\end{itemize}
\end{frame}

\begin{frame}
\centering
\Huge \textbf{Thank You!}
\end{frame}

\end{document}
